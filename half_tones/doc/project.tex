\documentclass[conference]{IEEEtran}
\usepackage{cite}
\usepackage[portuges,brazil]{babel}
\usepackage{amsmath,amssymb,amsfonts}
\usepackage{siunitx}
\usepackage{algorithmic}
\usepackage{graphicx}
\usepackage{textcomp}
\usepackage{hyperref}
\usepackage{listings}
\usepackage[toc,page]{appendix}
\usepackage[utf8]{inputenc}

\def\BibTeX{{\rm B\kern-.05em{\sc i\kern-.025em b}\kern-.08em
    T\kern-.1667em\lower.7ex\hbox{E}\kern-.125emX}}

\begin{document}

\title{Projeto 1: Aplicação de Técnicas de meio tons\\
\large .\\
\large MC920 - Introduçãao ao Processamento Digital de Imagem (MC920 / MO443) 2S2019\\
\large Professor: Hélio Pedrini}

\newcommand{\email}[1]{\href{mailto:#1}{#1}}

\author{
    \IEEEauthorblockN{Giovani Nascimento Pereira}
    \IEEEauthorblockA{
    \email{giovani.x.pereira@gmail.com} \\
    168609
    }
}

\maketitle

\section{Especificação do Problema}
As técnicas de meio-tons são empregadas para reduzir a quantidade de cores utilizadas para exibir uma imagem, mas procurando manter uma boa percepção visual por parte do usuário \cite{helio}.

Essa técnica foi empregada fazendo normalizando os valores de cada pixel e então distribuindo o erro (diferença entre o valor original do pixel e seu valor normalizado) de acordo com uma máscara especificada. Neste trabalho foram implementadas 6 abordagens de distribuições diferentes para esses erros:
\begin{itemize}
    \item Burkes
    \item Sierra
    \item Stevenson and Arce
    \item Stucki
    \item Floyd-Steinberg
    \item Jarvas Judice and Ninke
\end{itemize}

\section{Entrada de dados e execução do programa}
O script que faz a aplicação das técnicas de meio-tons foi desenvolvido em Python e tem como arquivo principal o \textbf{main.py}. Além disso, arquivos auxiliares foram criados para facilitar a divisão de responsabilidades dentre cada arquivo e deixar mais simples a compreensão de cada um individualmente.

Dentro da pasta \textit{Half-tones} temos o código que executa cada método de meio tom aplicado pelo script separadamente, e o arquivo \textbf{arguments.py} cuida dos parâmetros passados na execução do programa (foi feito usando a biblioteca \textit{argparse}).

A entrada esperada do script é uma imagem no formato PNG (\textit{imagem}), de dimensão qualquer. E para explicitar qual método de meio-tom será aplicado a imagem deve-se explicitar através de flags na execução do programa, como por exemplo \textit{-fs} para Floyd Steinberg ou \textit{-b} para Burke ou ainda \textit{-all} para todos os métodos implementados.

\begin{lstlisting}[language=bash]
$ python3 main.py [imagem] [flags]
\end{lstlisting}

Uma documentação pode ser exibida pelo programa com o comando:

\begin{lstlisting}[language=bash]
$ python3 main.py --help
\end{lstlisting}

Para exemplificação dos resultados foram adotadas as imagens disponibilizadas pelo professor em \url{https://www.ic.unicamp.br/~helio/imagens_coloridas/}.

 \section{Código e Decisões tomadas}

 \subsection{Pré processamento}
 A imagem que será trabalhada foi carregada usando \textbf{imageio}, ele já converte o arquivo PNG em uma matriz das dimensões da imagem onde cada posição da matriz representa um pixel e seu nível de cor. Contudo, o tipo de cada item dessa matriz é \textbf{uint8} que aceita números representados por até 8 bits (de 0 a 255).

 Contudo, esse intervalo de valores não é suficiente para fazer as operações matemáticas necessárias, nem mesmo aceita números negativos, e poderia causar overflow nos valores, o que traria resultados incorretos no processamento.

 Para solucionar esse problema, a matriz foi convertida em \textbf{int64} durante o processamento, e depois novamente convertida em \textbf{uint8} para salvar como arquivo PNG.

 \subsection{Direção da distribuição de erro}
 A implementação das distribuições de erro seguiram um padrão \textit{zigue-zague} ao longo das matrizes das imagens. Movendo em x de 0 a $x_{max}$ e depois de $x_{max}$ a 0, alternado, incrementando y. Isso, para cada banda de cor. Essa implementação está representada na imagem \ref{fig:dir}.
 Esse modelo segue para tentar reduzir efeitos de padrões indesejados ou direcionalidade que poderiam surgir na imagem de saída usando apenas uma movimentação unidirecional no eixo x \cite{helio}.

 \begin{figure}[ht]
     \centering
     \includegraphics[width=120pt]{direction.png}
     \caption{Esquema da direção que o algoritmo de distribuição de erro percorre a matriz da imagem}
     \label{fig:dir}
 \end{figure}

  \subsection{Implementação das técnicas de meio tom}
 Dado o algoritmo que percorre a matriz da imagem, a cada pixel era aplicado o algoritmo de distribuição de erro solicitado.

 Quando um ponto de distribuição de erro se encontrava em uma posição xyz que não pertencia à matriz (casos de borda), eles foram puramente desconsiderados.

 Cada técnica de meio-tom foi implementada em um arquivo separado com o mesmo nome da técnica empregada.

 \section{Resultados}

 \subsection{Imagens geradas}

 Executando o script para todas as distribuições implementadas e usando a imagem \textit{baboon.png} como entrada, foram geradas as imagens da figura \ref{fig:results}.

 \begin{figure}[ht]
     \centering
     \includegraphics[width=220]{baboons.png}
     \caption{Imagens geradas através do método de meios tons. (a) Imagem original, (b) Floyd-Steinberg, (c) Stevenson and Arce, (d) Burkes, (e) Sierra, (f) Stucki, (g) Jarvas Judice and Ninke}
     \label{fig:results}
 \end{figure}

 \section{Análise}
 \subsection{Considerações gerais}
 Durante a análise dos resultados das imagens obtidas, ficou evidente que o visualizador de imagens, bem como a dimensão com que a imagem era exibida tinha um impacto muito grande na percepção dos resultados de aplicação as técnicas de meio-tons \cite{srgb}.

 Na máquina onde o projeto foi desenvolvido os resultados se apresentavam como esperado, com canais de cor monocromáticos, quando os pixels eram observados usando o espaço de cor sRGB. Outros espaços de cor como P3 ou Adobe RGB mostravam resultados diferentes para as mesmas imagens analisadas, que não correspondiam ao esperado. E isso dependia da ferramenta de visualização utilizada.

 Dessa forma, toda a análise das cores resultantes pelos processos de meio-tons foram feitas observando através do espaço de cor sRGB.

 \subsection{Pontilhado}

 Se oservarmos uma imagem individualmente é possível notar que seus pixels, um a um, são formados exclusivamente por intensidades 0 e 255. A imagem \ref{fig:rgbc} mostra uma seção 4x4 pixels da imagem \ref{fig:results} (a), \textit{baboon} aplicada a distribuição de Floyd Steinberg.

 \begin{figure}[ht]
     \centering
     \includegraphics[width=130pt]{rgbc.png}
     \caption{Seção 4x4 da imagem \ref{fig:results} (a), \textit{baboon} aplicada a distribuição de Floyd Steinberg, com os valores das intensidades dos canais RGB em order representados dentro de cada abstração dos pixels.}
     \label{fig:rgbc}
 \end{figure}

 Esse era o resultado esperado, dessa forma, cada canal de cor, individualmente está monocromático. Mas mesmo assim, a percepção da imagem como um todo não é alterada. Uma pessoa com visão comum ainda é capaz de identificar o conteúdo da imagem, mesmo que ela tenha passado pela transformação de meios-tons.

 \subsection{Comparação dos Métodos de distribuição de erro}

 Usando como referência a imagem \textit{monalisa.png} disponibilizada pelo professor para a confecção do trabalho, foram executados os algoritmos de meio tons e extraídos alguns exemplos para análise. Os resultados são exibidos na figura \ref{fig:comparison}. A imagem original pode ser observada no \textit{Apêndice A} figura \ref{fig:monalisa}.

 \begin{figure}[ht]
     \centering
     \includegraphics[width=\linewidth]{kgbsc.png}
     \caption{Comparação entre as imagens resultantes e um corte 5x5 do canto superior esquerdo das imagens geradas. (a) Floyd-Steinberg, (b) Stevenson and Arce, (c) Burkes.}
     \label{fig:comparison}
 \end{figure}

 \section{Conclusão}
 Os resultados encontrados para as técnicas de meio-tons foram satisfatórios. Todas as imagens (seja em escala de cinza ou em cor) eram reconhecíveis após a aplicação de qualquer uma das técnicas de meio tom, e ao analisar os pixels das imagens resultantes, para cada banda de cor, a imagem havia se tornado monocromática (ou com nível 0 ou 255).

 Cada método de meio tom tem um resultado diferente, sendo que cada um pode ser mais recomendado dado o tipo de imagem que está sendo processada.

\begin{thebibliography}{00}

\bibitem{helio} Helio Pedrini. Trabalho 1. Introdução ao Processamento Digital de Imagem (MC920 / MO443), 2019.\\

\bibitem{srgb} SRGB. disponível em \url{https://en.wikipedia.org/wiki/SRGB}. Acesso 11 de setembro de 2019.

\end{thebibliography}

\newpage
\newpage
\begin{appendices}
\chapter{Apêndice A - Imagens}
 \begin{figure}[ht]
     \centering
     \includegraphics[width=\linewidth]{monalisa-path.png}
     \caption{Imagem de teste \textit{monalisa.png} original, com corte 5x5 do canto superior esquerdo da imagens.}
     \label{fig:monalisa}
 \end{figure}

\end{appendices}

\end{document}
