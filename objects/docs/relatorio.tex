\documentclass[conference]{IEEEtran}
\usepackage{cite}
\usepackage[portuges,brazil]{babel}
\usepackage{amsmath,amssymb,amsfonts}
\usepackage{siunitx}
\usepackage{algorithmic}
\usepackage{graphicx}
\usepackage{textcomp}
\usepackage{hyperref}
\usepackage{listings}
\usepackage[toc,page]{appendix}
\usepackage[utf8]{inputenc}

\def\BibTeX{{\rm B\kern-.05em{\sc i\kern-.025em b}\kern-.08em
    T\kern-.1667em\lower.7ex\hbox{E}\kern-.125emX}}

\begin{document}

\title{Projeto 3: Detecção de Objetos\\
\large .\\
\large MC920 - Introdução ao Processamento Digital de Imagem (MC920 / MO443) 2S2019\\
\large Professor: Hélio Pedrini}

\newcommand{\email}[1]{\href{mailto:#1}{#1}}

\author{
    \IEEEauthorblockN{Giovani Nascimento Pereira}
    \IEEEauthorblockA{
    \email{giovani.x.pereira@gmail.com} \\
    168609
    }
}

\maketitle

\section{Definição do problema}

  Métodos de detecção de objetos em imagens tem inúmeras aplicações no cotidiano e em processos de tratamentos de imagens.

  Tais métodos podem ser utilizados para identificar a existência de objetos conhecidos, e com isso, encontrar outras métricas como dimensão, posição, forma, etc.

    \subsection{Objetivo}

        O objetivo deste projeto é implementar métodos de detecção e descrição de objetos em imagens coloridas (.png), e analisar os resultados obtidos, encontrando múlitplos objetos por imagem.

    \subsection{Execução do projeto}

        O projeto pode ser executado utilizando o comando:

        \begin{lstlisting}[language=bash]
        $ python3 main.py [imagem]
        \end{lstlisting}

        Onde o parâmetro \textit{[imagem]} define o caminho relativo da imagem que será processada.

        Ao executar o projeto, ele irá aplicar as tranformações disponíveis e salvar as imagens resultantes na pasta \textit{/out}.

        As aplicações são:

        \begin{itemize}
            \item Escala de cinza
            \item Detecção de bordas
            \item Identificação dos objetos (label)
        \end{itemize}

        Além disso, será impresso na saída padrão dados de todos os objetos encontrados das imagens:

        \begin{itemize}
            \item Área
            \item Perímetro
            \item Excentricidade
            \item Solidez
        \end{itemize}

        E para cada imagem, será montado um histograma que mede a quantidade de objetos por área da imagem.

\section{Transformações}

    \subsection{Conversão em escala de cinza}

        A imagens utilizadas estavam em formato \textit{.png}, que são essencialmente imagens com 3 canais de cor, ou seja, coloridas.

        Parte da análise era feita em imagens em escala de cinza, para isso, foi utilizada a biblioteca \textit{scikit-image}, com o módulo \textit{colors} que possui o método \textit{rgb2gray} que converte uma imagem colorida em escala de cinza.

        Isso também implica que a entrada desse método é um array tridimensional (3D) e a saída um array bidimensional (2D).

        Um efeito colateral da aplicação deste método é que ele retorna uma matriz de valores fracionários (float) entre 0 e 1. Para salvar a imagem, ela deve ser convertida novamente em uma matriz de inteiros, onde a variação fica entre 0 e 255 (256 níveis de cinza discretizados).

        A própria biblioteca trata esse problema ao salvar a imagem.

        Podemos vez um exemplo de conversão em escala de cinza na imagem \ref{cinza}.

        \begin{figure}[ht]
            \centering
            \includegraphics[width=200]{grayscale.png}
            \caption{Imagem gerada através da aplicação da conversão da imagem original colorida em escala de cinza.}
            \label{cinza}
        \end{figure}

    \subsection{Detecção de bordas}

        Para a detecção de bordas, foi utilizado o método de \textit{Sobel} disponível na biblioteca \textit{scikit-image}.
        Para tanto, a imegam deve ser previamente convertida em escala de cinza, de forma a manter apenas um canal de cor, ao invés de 3 canais. Esse processo foi feito utilizando também a biblioteca \textit{scikit-image}, através do método \textit{rgbToGray}.

        Ao aplicar o método de \textit{Sobel} na imagem desejada, ele destaca as bordas das imagens e remove toda informação adicional da imagem.

        Um exemplo da aplicação do Método de detecção de bordas de Sobel pode ser observado na imagem \ref{sobel}.

        \begin{figure}[ht]
            \centering
            \includegraphics[width=200]{edges.png}
            \caption{Imagem gerada através da aplicação do método de sobel para identificação de bordas.}
            \label{sobel}
        \end{figure}

    \subsection{Identificação dos Objetos}

        Para a identificação dos objetos foi utilizada a biblioteca \textit{scikit-image}, mais específicamente o módulo \textit{measure}.

        Este módulo tráz o método \textit{regionprops}, que funciona tanto em imagens em escala de cinza (2D) como em imagens coloridas (3D).

        No nosso caso, foi utilizado apenas para imagens em escalas de cinza (pois isso facilitava outras análises e medições da mesma biblioteca).

        Este método retorna uma imagem onde os objetos encontrados são coloridos inteiramente e com cores diferentes entre si. Por exemplo, 2 objetos distinto teriam seus pixels inteiramente formados por apenas um valor de níveis de cinza, e seriam diferentes entre os objetos em si.

        Um exemplo de aplicação pode ser observado na \imagem ref{labeliza}.

        \begin{figure}[ht]
            \centering
            \includegraphics[width=200]{labeled.png}
            \caption{Imagem gerada através da aplicação da geração de labels numa imagem em escala de cinza.}
            \label{labeliza}
        \end{figure}

\section{Resultados}

    Todas as transformações citadas foram aplicadas para as três imagens disponibilizadas de entrada, e os resultados podem ser vistos a seguir.

    Isso inclui os histogramas que comparam as áreas dos objetos encontrados nas imagens.


    \begin{figure}[ht]
        \centering
        \includegraphics[width=\linewidth]{img1.png}
        \caption{a) imagem original (arquivo de entrada \textit{img01.png}). b) transformação em escala de cinza. c) Detecção de bordas. d) Labels.}
        \label{img1}
    \end{figure}

    \begin{figure}[ht]
        \centering
        \includegraphics[width=200]{hist1.png}
        \caption{Histograma da distribuição dos objetos por área encontrados na imagem \textit{img01.png}.}
        \label{hist1}
    \end{figure}

    \begin{figure}[ht]
        \centering
        \includegraphics[width=\linewidth]{img2.png}
        \caption{a) imagem original (arquivo de entrada \textit{img02.png}). b) transformação em escala de cinza. c) Detecção de bordas. d) Labels.}
        \label{img2}
    \end{figure}

    \begin{figure}[ht]
        \centering
        \includegraphics[width=200]{hist2.png}
        \caption{Histograma da distribuição dos objetos por área encontrados na imagem \textit{img02.png}.}
        \label{hist2}
    \end{figure}

\section{Análises}

    A detecção de objetos encontrou o número correto de objetos nas imagens, é possível perceber isso nas figuras \ref{img1} e \ref{img2} itens (d), que mostram todos os objetos encontrados - e é fácil perceber que todos os objetos existentes nas imagens em (a), as imagens originais, foram todos detectados.

    Os histogramas das imagens \ref{hist1} e \ref{hist2} trazem a quantidade de objetos por área da imagem, onde um objeto é considerado pequeno se sua área é menor que $1500$ pixels, médio se entre $1500$ e $3000$ pixels, e grande se maior que $3000$ pixels.


\section{Conclusão}

    Com os dados obtidos dos métodos de detecção e descrição de objetos podemos dizer que o resultado do projeto foi satisfatório.

    A detecção de bordas foi muito bem comportada para o conjunto de imagens analisados, conseguindo destacar fielmente as bordas dos objetos visualmente presentes nas imagens.

    A transformação em escala de cinza também foi dentro do esperado, o conjunto resultante passou a ser uma matriz bidimensional (2D), ou seja, com apenas um canal de cor, sendo assim monocromática.

    A detecção de objetos também foi dentro do esperado, o número de objetos detectados foi o correto e as propriedades puderam sex extraídas de cada objeto.

\begin{thebibliography}{00}

  \bibitem{helio} Helio Pedrini. Trabalho 3. Introdução ao Processamento Digital de Imagem (MC920 / MO443), 2019.\\

  \bibitem{srgb} SRGB. disponível em \url{https://en.wikipedia.org/wiki/SRGB}. Acesso 11 de setembro de 2019.

\end{thebibliography}

\end{document}
