\documentclass[conference]{IEEEtran}
\usepackage{cite}
\usepackage[portuges,brazil]{babel}
\usepackage{amsmath,amssymb,amsfonts}
\usepackage{siunitx}
\usepackage{algorithmic}
\usepackage{graphicx}
\usepackage{textcomp}
\usepackage{hyperref}
\usepackage{listings}
\usepackage[toc,page]{appendix}
\usepackage[utf8]{inputenc}

\def\BibTeX{{\rm B\kern-.05em{\sc i\kern-.025em b}\kern-.08em
    T\kern-.1667em\lower.7ex\hbox{E}\kern-.125emX}}

\begin{document}

\title{Projeto 3: Detecção de Objetos\\
\large .\\
\large MC920 - Introdução ao Processamento Digital de Imagem (MC920 / MO443) 2S2019\\
\large Professor: Hélio Pedrini}

\newcommand{\email}[1]{\href{mailto:#1}{#1}}

\author{
    \IEEEauthorblockN{Giovani Nascimento Pereira}
    \IEEEauthorblockA{
    \email{giovani.x.pereira@gmail.com} \\
    168609
    }
}

\maketitle

\section{Definição do problema}

  Métodos de detecção de objetos em imagens tem inúmeras aplicações no cotidiano e em processos de tratamentos de imagens.

  Tais métodos podem ser utilizados para identificar a existência de objetos conhecidos, e com isso, encontrar outras métricas como dimensão, posição, forma, etc.

    \subsection{Objetivo}

        O objetivo deste projeto é implementar métodos de detecção e descrição de objetos em imagens coloridas (.png), e analisar os resultados obtidos, encontrando múlitplos objetos por imagem.

    \subsection{Execução do projeto}

        O projeto pode ser executado utilizando o comando:

        \begin{lstlisting}[language=bash]
        $ python3 main.py [imagem]
        \end{lstlisting}

        Onde o parâmetro \textit{[imagem]} define o caminho relativo da imagem que será processada.

    \subsection{Dependências}


\section{Transformações}

    \subsection{Detecção de bordas}

        Para a detecção de bordas, foi utilizado o método de \textit{Sobel} disponível na biblioteca \textit{scikit-image}.
        Para tanto, a imegam deve ser previamente convertida em escala de cinza, de forma a manter apenas um canal de cor, ao invés de 3 canais. Esse processo foi feito utilizando também a biblioteca \textit{scikit-image}, através do método \textit{rgbToGray}.

        Ao aplicar o método de \textit{Sobel} na imagem desejada, ele destaca as bordas das imagens e remove toda informação adicional da imagem.

\section{Resultados}


\section{Análises}


\section{Conclusão}

    Com os dados obtidos dos métodos de detecção e descrição de objetos podemos dizer que o resultado do projeto foi satisfatório.

    A detecção de bordas foi muito bem comportada para o conjunto de imagens analisados, conseguindo destacar fielmente as bordas dos objetos visualmente presentes nas imagens.



\begin{thebibliography}{00}

  \bibitem{helio} Helio Pedrini. Trabalho 3. Introdução ao Processamento Digital de Imagem (MC920 / MO443), 2019.\\

  \bibitem{srgb} SRGB. disponível em \url{https://en.wikipedia.org/wiki/SRGB}. Acesso 11 de setembro de 2019.

\end{thebibliography}

\end{document}
