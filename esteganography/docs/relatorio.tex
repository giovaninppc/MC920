\documentclass[conference]{IEEEtran}
\usepackage{cite}
\usepackage[portuges,brazil]{babel}
\usepackage{amsmath,amssymb,amsfonts}
\usepackage{siunitx}
\usepackage{algorithmic}
\usepackage{graphicx}
\usepackage{textcomp}
\usepackage{hyperref}
\usepackage{listings}
\usepackage[toc,page]{appendix}
\usepackage[utf8]{inputenc}

\def\BibTeX{{\rm B\kern-.05em{\sc i\kern-.025em b}\kern-.08em
    T\kern-.1667em\lower.7ex\hbox{E}\kern-.125emX}}

\begin{document}

\title{Projeto 4: Esteganografia\\
\large .\\
\large MC920 - Introdução ao Processamento Digital de Imagem (MC920 / MO443) 2S2019\\
\large Professor: Hélio Pedrini}

\newcommand{\email}[1]{\href{mailto:#1}{#1}}

\author{
    \IEEEauthorblockN{Giovani Nascimento Pereira}
    \IEEEauthorblockA{
    \email{giovani.x.pereira@gmail.com} \\
    168609
    }
}

\maketitle

\section{Definição do problema}


    \subsection{Objetivo}

    \subsection{Execução do projeto}



\section {Codificação}

\section {Decodificação}

\section{Resultados}



\section{Análises}



\section{Conclusão}


\begin{thebibliography}{00}

  \bibitem{helio} Helio Pedrini. Trabalho 4. Introdução ao Processamento Digital de Imagem (MC920 / MO443), 2019.\\

  \bibitem{srgb} SRGB. disponível em \url{https://en.wikipedia.org/wiki/SRGB}. Acesso 11 de setembro de 2019.

\end{thebibliography}

\end{document}
